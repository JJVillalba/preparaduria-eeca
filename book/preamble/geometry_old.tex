%--------------------------------------------------------------------------------
%	PAGE LAYOUT
%--------------------------------------------------------------------------------

% Define lengths to set the scale of the document. Changing these 
% lengths should affect all the other pagesize-dependent elements in the 
% layout, such as the geometry of the page, the spacing between 
% paragraphs, and so on. (As of now, not all the elements rely on hscale 
% and vscale; future work will address this shortcoming.)
\newlength{\hscale}
\newlength{\vscale}

% By default, the scales are set to work for a4paper
\setlength{\hscale}{1mm}
\setlength{\vscale}{1mm}

\usepackage[
	paperwidth=210\hscale,
	paperheight=297\vscale,
]{geometry}


% Command to choose among the three possible layouts
\DeclareDocumentCommand{\pagelayout}{m}{%
	\ifthenelse{\equal{margin}{#1}}{\marginlayout\marginfloatsetup}{}%
	\ifthenelse{\equal{wide}{#1}}{\widelayout\widefloatsetup}{}%
	\ifthenelse{\equal{fullwidth}{#1}}{\fullwidthlayout\widefloatsetup}{}%
}

\newif\ifwidelayout%
\def\IfWideLayout{%
	\ifwidelayout%
		\expandafter\@firstoftwo%
	\else%
		\expandafter\@secondoftwo%
	\fi%
}

% Layout #1: large margins
\newcommand{\marginlayout}{%
	\newgeometry{
		top=27.4\vscale,
		bottom=27.4\vscale,
		inner=24.8\hscale,
		textwidth=107\hscale,
		marginparsep=6.2\hscale,
		marginparwidth=47.7\hscale,
	}%
	\recalchead%
	\widelayoutfalse%
}

% Layout #2: small, symmetric margins
\newcommand{\widelayout}{%
	\newgeometry{
		top=27.4\vscale,
		bottom=27.4\vscale,
		inner=24.8\hscale,
		outer=24.8\hscale,
		marginparsep=0mm,
		marginparwidth=0mm,
	}%
	\recalchead%
	\widelayouttrue%
}

% Layout #3: no margins and no space above or below the body
\newcommand{\fullwidthpage}{%
	\newgeometry{
		top=0mm,
		bottom=0mm,
		inner=0mm,
		outer=0mm,
		marginparwidth=0mm,
		marginparsep=0mm,
	}%
	\recalchead%
	\widelayouttrue%
}

% Set the default page layout
\AtBeginDocument{\pagelayout{margin}}

%----------------------------------------------------------------------------------------
%	HEADERS AND FOOTERS
%----------------------------------------------------------------------------------------

\usepackage{scrlayer-scrpage}		% Customise head and foot regions

% Set the header height to prevent a warning
%\setlength{\headheight}{27.4\vscale}
% Increase the space between header and text
\setlength{\headsep}{11\vscale}

% Define some LaTeX lengths used in the page headers
\newlength{\headtextwidth} % This is the width of the text
\newlength{\headmarginparsep} % This is the width of the whitespace between text and margin
\newlength{\headmarginparwidth} % This is the width of the margin
\newlength{\headtotal} % This is the total width of the header
\newlength{\contentwidth} % An alias for \headtotal
\newcommand{\recalchead}{% Command to recalculate the header-related length when needed
	\setlength{\headtextwidth}{\textwidth}%
	\setlength{\headmarginparsep}{\marginparsep}%
	\setlength{\headmarginparwidth}{\marginparwidth}%
	\setlength{\headtotal}{\headtextwidth+\headmarginparsep+\headmarginparwidth}%
	\setlength{\contentwidth}{\headtotal}%
}

\AtBeginDocument{% Recalculate the header-related lengths
	\recalchead%
}

% Header style with chapter number, chapter title, and page in the header (used throughout the document)
\renewpagestyle{scrheadings}{%
	{\smash{\hspace{-\headmarginparwidth}\hspace{-\headmarginparsep}\makebox[\headtotal][l]{%
		\makebox[7\hscale][r]{\thepage}%
		\makebox[3\hscale]{}\rule[-1mm]{0.5pt}{19\vscale-1mm}\makebox[3\hscale]{}%
		\makebox[\headtextwidth][l]{\leftmark}}}}%
	{\smash{\makebox[0pt][l]{\makebox[\headtotal][r]{%
		\makebox[\headtextwidth][r]{\hfill\rightmark}%
		\makebox[3\hscale]{}\rule[-1mm]{0.5pt}{19\vscale-1mm}\makebox[3\hscale]{}%
		\makebox[7\hscale][l]{\thepage}}}}}%
	{\smash{\makebox[0pt][l]{\makebox[\headtotal][r]{%
		\makebox[\headtextwidth][r]{\hfill\rightmark}%
		\makebox[3\hscale]{}\rule[-1mm]{0.5pt}{19\vscale-1mm}\makebox[3\hscale]{}%
		\makebox[7\hscale][l]{\thepage}}}}}%
}{%
	{}%
	{}%
	{}%
}

% Header style with neither header nor footer (used for special pages)
\renewpagestyle{plain.scrheadings}{%
	{}%
	{}%
	{}%
}{%
	{}%
	{}%
	{}%
}

% Header style with an empty header and the page number in the footer
\newpagestyle{pagenum.scrheadings}{%
	{}%
	{}%
	{}%
}{%
	{\makebox[\textwidth][r]{\thepage}}%
	{\makebox[\textwidth][l]{\thepage}}%
	{\makebox[\textwidth][l]{\thepage}}%
}

% Header style with an empty header and the page number in the footer
\newpagestyle{centeredpagenum.scrheadings}{%
	{}%
	{}%
	{}%
}{%
	{\hspace{-\headmarginparwidth}\hspace{-\headmarginparsep}\makebox[\headtotal][l]{\hfill\thepage\hfill}}
	{\makebox[0pt][l]{\makebox[\headtotal][r]{\hfill\thepage\hfill}}}%
	{\makebox[0pt][l]{\makebox[\headtotal][r]{\hfill\thepage\hfill}}}%
}

% Command to print a blank page
\newcommand\blankpage{%
	\null%
	\thispagestyle{empty}%
	\newpage%
}

% Set the default page style
\pagestyle{plain.scrheadings}



%-------------------------------------------------------------------------------
%	FRONT-, MAIN-, BACK- MATTERS BEHAVIOUR
%-------------------------------------------------------------------------------

% Front matter
\let\oldfrontmatter\frontmatter % Store the old command
\renewcommand{\frontmatter}{%
	\oldfrontmatter% First of all, call the old command
	\pagestyle{plain.scrheadings}% Use a plain style for the header and the footer
	\pagelayout{wide}% Use a wide page layout
	\setchapterstyle{plain} % Choose the default chapter heading style
	% \sloppy % Required to better break long lines
}

%------------------------------------------------

% Main matter
\let\oldmainmatter\mainmatter % Store the old command
\renewcommand{\mainmatter}{%
	\oldmainmatter% Call the old command
	\pagestyle{scrheadings}% Use a fancy style for the header and the footer
	\pagelayout{margin}% Use a 1.5 column layout
	\setchapterstyle{kao} % Choose the default chapter heading style
}

%------------------------------------------------

% Appendix
\let\oldappendix\appendix% Store the old command
\renewcommand{\appendix}{%
	\oldappendix% Call the old command
	\bookmarksetup{startatroot}% Reset the bookmark depth
}

%------------------------------------------------

% Back matter
\let\oldbackmatter\backmatter% Store the old command
\renewcommand{\backmatter}{%
	\oldbackmatter% Call the old command
	\bookmarksetup{startatroot}% Reset the bookmark depth
	\pagestyle{plain.scrheadings}% Use a plain style for the header and the footer
	\pagelayout{wide}% Use a wide page layout
	\setchapterstyle{plain} % Choose the default chapter heading style
}

\usepackage[hypcap=true]{caption} % Correctly placed anchors for hyperlinks
\usepackage{floatrow} % Set up captions of float
\usepackage{floatrow} 

% Captions for the 'margin' layout
\NewDocumentCommand{\marginfloatsetup}{}{%
	\floatsetup[figure]{% Captions for figures
		margins=hangright,% Put captions in the margins
		facing=yes,%
		capposition=beside,%
		capbesideposition={bottom,right},%
		capbesideframe=yes,%
		capbesidewidth=\marginparwidth,% Width of the caption equal to the width of the margin
		capbesidesep=marginparsep,%
		floatwidth=\textwidth,% Width of the figure equal to the width of the text
	}%
	\floatsetup[widefigure]{% Captions for wide figures
		margins=hangright,% Put captions below the figure
		facing=no,%
		capposition=bottom%
	}%
	\floatsetup[table]{% Captions for tables
		margins=hangright,% Put captions in the margin
		facing=yes,%
		capposition=beside,%
		capbesideposition={top,right},%
		%capbesideposition=outside,
		capbesideframe=yes,%
		capbesidewidth=\marginparwidth,% Width of the caption equal to the width of the margin
		capbesidesep=marginparsep,%
		floatwidth=\textwidth,% Width of the figure equal to the width of the text
	}%
	\floatsetup[widetable]{% Captions for wide tables
		margins=hangright,% Put captions above the table
		facing=no,%
		capposition=above%
	}%
	\floatsetup[longtable]{% Captions for longtables
		margins=raggedright,% Overwrite the hangright setting from the `table' environment
		%LTcapwidth=table,% Set the width of the caption equal to the table's
	}%
	\floatsetup[lstlisting]{% Captions for lstlisting
		margins=hangright,% Put captions in the margin
		facing=yes,%
		capposition=beside,%
		capbesideposition={top,right},%
		%capbesideposition=outside,
		capbesideframe=yes,%
		capbesidewidth=\marginparwidth,% Width of the caption equal to the width of the margin
		capbesidesep=marginparsep,%
		floatwidth=\textwidth,% Width of the figure equal to the width of the text
	}%
	\floatsetup[listing]{% Captions for listing (minted package)
		margins=hangright,% Put captions in the margin
		facing=yes,%
		capposition=beside,%
		capbesideposition={top,right},%
		%capbesideposition=outside,
		capbesideframe=yes,%
		capbesidewidth=\marginparwidth,% Width of the caption equal to the width of the margin
		capbesidesep=marginparsep,%
		floatwidth=\textwidth,% Width of the figure equal to the width of the text
	}%
	%\captionsetup*[lstlisting]{%
	%	format=llap,%
	%	labelsep=space,%
	%	singlelinecheck=no,%
	%	belowskip=-0.6cm,%
	%}%
}
% Captions for the 'wide' layout
\NewDocumentCommand{\widefloatsetup}{}{%
	\floatsetup[figure]{ % Captions for figures
		capposition=bottom,%
		margins=centering,%
		floatwidth=\textwidth%
	}
	\floatsetup[widefigure]{ % Captions for wide figures
		margins=hangoutside, % Put captions below the figure
		facing=yes,%
		capposition=bottom%
	}
	\floatsetup[table]{ % Captions for tables
		capposition=above,%
		margins=centering,%
		floatwidth=\textwidth%
	}
	\floatsetup[widetable]{ % Captions for wide tables
		margins=hangoutside, % Put captions above the table
		facing=yes,%
		capposition=above%
	}
	\floatsetup[lstlisting]{ % Captions for lstlistings
		capposition=above,%
		margins=centering,%
		floatwidth=\textwidth%
	}
	\floatsetup[listing]{ % Captions for listings (minted package)
		capposition=above,%
		margins=centering,%
		floatwidth=\textwidth%
	}
	\captionsetup*[lstlisting]{% Captions style for lstlistings
		%format=margin,%
		labelsep=colon,%
		strut=no,%
		singlelinecheck=false,%
		indention=0pt,%
		parindent=0pt,%
		aboveskip=6pt,%
		belowskip=6pt,%
		belowskip=-0.1cm%
	}
}
